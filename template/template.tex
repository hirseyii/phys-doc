%%%%%%%%%% BEGIN PREAMBLE %%%%%%%%%%
\documentclass[11pt,oneside]{book}
%%%%% PACKAGES %%%%%
\usepackage[top=2.54cm, bottom=2.54cm, left=3cm, right=2cm]{geometry}
\usepackage{graphicx}
\usepackage{lipsum}
\usepackage{amsmath}
\usepackage{amssymb}
\usepackage[font={scriptsize}]{caption}
\usepackage{float} 
\usepackage{changepage}
\usepackage{titlesec}

\titleformat{\chapter}[display]
	{\normalfont\huge\bfseries}{\chaptertitlename\ \thechapter}{20pt}{\Huge}
\titlespacing*{\chapter}{-.5in}{.5pt}{40pt}

%%%%% END PREABMLE %%%%%%
\begin{document}
\chapter{This is the chapter title}
INSERT CONTENTS AND STUFF HERE
\begin{enumerate}
	\item stuff
	\item more stuff
	\item etc.
	\item stuuuffff
\end{enumerate}
\pagebreak
As you can see the chapter title is offset from the text, if you wish to change the margins at all before finalise the template just do that and then do a pull request.
%
\section{Equation formatting options}
Section heading is not offset, perhaps we could change that, it's up to you?
 There are several different options for equations, I think there are 3 major things to think about:
%
\begin{enumerate}
	\item Vectors
	\item alignment
	\item problems?
\end{enumerate}
%
\subsection{1 - Vectors}
The options for vector are really limited to bold or underlining, here are some examples I also included the arrow on the top but i'm not sure it looks too great:
%
$$\mathbf{\nabla}\times\mathbf{E} = \mathbf{\dot{B}}$$
%
$$\pmb{\nabla}\times\pmb{E} = \pmb{\dot{B}}$$
%
$$\underline{\nabla}\times\underline{E} = \underline{\dot{B}}$$
%
$$\vec{\nabla}\times\vec{E} = \vec{\dot{B}}$$
%
\subsection{2 - alignment}
This is really concerning lots of equation in a row for a long derivation, i'll give 2 examples and see what you think.
%
$$\frac{1}{2}mv^2=\frac{3}{2}K_BT + A - A$$
%
$$v^2=\frac{3K_BT}{m}$$
%
$$v=\sqrt{\frac{3K_BT}{m}}$$
%
In that example the equations are all aligned in the middle
%
\begin{align*}	
	\frac{1}{2}mv^2&=\frac{3}{2}K_BT + A - A\\
	v^2&=\frac{3K_BT}{m}\\
	v&=\sqrt{\frac{3K_BT}{m}}\\
\end{align*}
%
Now all the equations are aligned at the equals sign, there's arguments for both but generally I think the first one looks more natural but its up to you.
%
\subsection{3 - problems}
For problems and examples I want to copy the Griffiths book, I'll show below.\\
%
\noindent\rule[0.2ex]{\linewidth}{0.1cm}
\subsubsection{Examples 1.1}
\begin{adjustwidth}{1cm}{1cm}
	\begin{enumerate}
		\item Using maths, complete the following
		\begin{enumerate}
			\item Derive time
			\item Show that the soul survives after death
		\end{enumerate}
		\item By considering the twin paradox show:
		\begin{enumerate}
			\item Special relativity is wrong
			\item Kill yourself
		\end{enumerate}
	\end{enumerate}
\end{adjustwidth}
\noindent\rule[0.5ex]{\linewidth}{0.1cm}
%
\end{document}