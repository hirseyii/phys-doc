\chapter{Introduction to Photonics}
\minitoc
\pagebreak
\section{Measurement of light's power}
<<<<<<< HEAD
There are two main ways of quantifying an electromagnetic waves power: radiometric and photometric.
Photometric is how much power will be seen by the human eye, while radiometric is a direct measurement of the waves power.
In these cases, light is defined as being in the visible spectrum although radiometry covers ultraviolet and infrared radiation.
The photometric measurement uses a standard responsivity of the human eye which is defined by the $V(\lambda)$ curve  shown in figure \ref{fig:ItP:eye}.
\begin{figure}[H]
	\includegraphics[width=\linewidth]{eye}
	\caption{The graph shows the curve for the photopic response, the black line, and the scotopic response, the green line, of a standard human eye}
	\label{fig:ItP:eye}
=======
There are two main ways of quantifying an the power of an electromagnetic wave: radiometric, and photometric.
 Photometric is the amount of power seen by the human eye, while radiometric is a direct measurement of the wave's power.
 In these cases, light is defined as being in the visible spectrum although radiometry covers ultraviolet and infrared radiation.
 The photometric measurement uses the standard responsivity of the human eye which is defined by the $V(\lambda)$ curve  shown in figure \ref{fig:eye}.
%
\begin{figure}[H]
	\includegraphics[width=\linewidth]{eye}
	\caption{The graph shows curves of the photopic response, the black line, and the scotopic response, the green line, for a typical human eye.}
	\label{fig:eye}
>>>>>>> 4039825f40075fbfad7e72f814cd199f192097fd
\end{figure}
%
There are two main types of responses: the scotopic (the green line) for an eye adapted to the dark, and the photopic (the black line) for an eye adapted to bright light.
 The radiant flux spectral distribution, $\phi_e(\lambda)$, can be used to calculate the (photometric) luminous flux, $\phi_v$ using
%
\begin{equation}
	\phi_v = K\int{V(\lambda)\phi_e(\lambda)d\lambda},
	\label{eq:luminous}
\end{equation}
%
where $K$ is the spectral luminous efficiency defined as $683$ lmW$^{-1}$.
\section{Polarisation}
<<<<<<< HEAD
For the most part the basics of polarisation are the same as in \textit{Wave Optics} although the definition of circularly polarised light may be different.
 For example, right hand circularly polarised light is defined as when you look towards the source with light coming towards you, you can curl your right hand in the direction of the rotation with the thumb pointing towards you. %This is terrible, I'm sorry
 The same applies for left hand circularly polarised light, although with the left hand.
 S- and p-polarised light mean the same thing as they do in \textit{Wave Optics}.
 \\
 The intensity reflection coefficient, $R$, for incident light, $\theta_i$ = 0, is
 \begin{equation}
 R = \left(\frac{n_1-n_2}{n_1+n_2}\right)^2
 \end{equation}
 and the 
 \[x=y^5\]
 $$x=y^5$$
=======
%
%
%
For the most part, the basics of polarisation are the same as in \textit{Wave Optics}, although the definition of the handedness of circularly polarised light may be different.
 For example, right circularly polarised light is defined as follows: looking towards the source with light coming towards the observer, you can curl your right hand in the direction of the rotation, with the thumb pointing towards you.
 The same applies for left circularly polarised light, except using the left hand.
\par
The intensity reflection coefficient, $R$, for incident light, $\theta_i = 0$, is
\begin{equation}
	R = \left(\frac{n_1-n_2}{n_1+n_2}\right)^2
\end{equation}
>>>>>>> 4039825f40075fbfad7e72f814cd199f192097fd
 


