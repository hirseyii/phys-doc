\chapter{Astrophysical Processes}
\minitoc
\pagebreak
\section{Quantifying radiation}
\subsection{Flux \& intensity}
Recalling from \textit{Introduction to Astrophysics \& Cosmology} we define the energy flux density (or simply flux), $F$ as the power per unit surface area.
%
$$ \diff E = F \diff A \diff t $$
%
Note that flux is dependant on the orientation of the unit surface with respect to the Poynting vector of the radiation field.
 Flux has units of J/s/m$^2$.
 Often it is more useful to use the \emph{monochromatic} energy flux density, $F_\nu$ where
%
$$ \diff E = F_\nu \diff A \diff t \diff \nu. $$
%
This gives the energy flux per unit of bandwidth, (frequency interval of radiation) which has units J/s/m$^2$/Hz.
\par 
In order to account for the differing directions of photons, the flux is further generalised by defining the specific monochromatic intensity, $I_\nu$.
 This quantifies the power per unit frequency from a range of directions covering a solid angle element $\diff \Omega$ given by
%
$$ \diff E = I_\nu \diff A dt \diff \nu \diff \Omega. $$
%
This then has units J/s/m$^2$/Hz/sr.
 Unlike flux, the intensity is constant along the direction of propagation of photons (in free space).
\par
From the respective definitions, we can relate flux and intensity by
%
\begin{align*}
	F_\nu \diff A &= \int \mathbf{I_\nu} \dotp \mathbf{\diff A} \diff \Omega
	\linedrop
	F_\nu \diff A &= \int I_\nu \cos(\theta) \diff A \diff \Omega
	\linedrop
	\therefore F_\nu &= \int I_\nu \cos(\theta) \diff \Omega
\end{align*}
%

\subsubsection{Solid angles: an aside}
The solid angle element $\diff \Omega$ is defined
%
$$ \diff \Omega = sin(\theta) \diff \theta \diff \phi , $$
%
hence when integrated over a complete sphere, $\Omega = 4\pi$~sr.
 This corresponds to the area of a sphere with unit radius.
 Considering the fraction of the total surface area, $A$ of a sphere with radius $R$, we find that
%
\begin{align*}
	\frac{A}{4\pi R^2} &= \frac{\Omega}{4\pi}
	\linedrop
	\Omega &= \frac{A}{R^2}.
\end{align*}
%
As asserted previously, the solid angle corresponds to the area on a sphere with unit radius.
%
\subsubsection{Surface Brightness}
Consider an object which a circular region in the sky with radius $\theta_c$.
 We choose to position this object such that it is located at the pole of the coordinate system.
 This is done for convenience such that the setup is rotationally symmetric in $\phi$.
 We can then calculate the flux incident on an aperture.
%
\begin{align}
	F_\nu &= 2\pi I\nu \int_{0}^{\theta_c}\diff\theta \sin(\theta) \cos(\theta)
\end{align}

 
 
 
 
 
 
 
 
 
 
 
 