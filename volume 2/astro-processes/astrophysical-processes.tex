\chapter{Astrophysical Processes}
\minitoc
\pagebreak
\section{Quantifying radiation}
\subsection{Flux \& intensity}
Recalling from \textit{Introduction to Astrophysics \& Cosmology} we define the energy flux density (or simply flux), $F$ as the power per unit surface area.
%
$$ \diff E = F \diff A \diff t $$
%
Note that flux is dependant on the orientation of the unit surface with respect to the Poynting vector of the radiation field.
 Flux has units of J/s/m$^2$.
 Often it is more useful to use the \emph{monochromatic} energy flux density, $F_\nu$ where
%
$$ \diff E = F_\nu \diff A \diff t \diff \nu. $$
%
This gives the energy flux per unit of bandwidth, (frequency interval of radiation) which has units J/s/m$^2$/Hz.
\par 
In order to account for the differing directions of photons, the flux is further generalised by defining the specific monochromatic intensity, $I_\nu$.
 This quantifies the power per unit frequency from a range of directions covering a solid angle element $\diff \Omega$ given by
%
$$ \diff E = I_\nu \diff A dt \diff \nu \diff \Omega. $$
%
This then has units J/s/m$^2$/Hz/sr.
 Unlike flux, the intensity is constant along the direction of propagation of photons (in free space).
\par
From the respective definitions, we can relate flux and intensity by
%
\begin{align*}
	F_\nu \diff A &= \int \mathbf{I_\nu} \dotp \mathbf{\diff A} \diff \Omega
	\linedrop
	F_\nu \diff A &= \int I_\nu \cos(\theta) \diff A \diff \Omega
	\linedrop
	\therefore F_\nu &= \int I_\nu \cos(\theta) \diff \Omega
\end{align*}
%

\subsubsection{Solid angles: an aside}
The solid angle element $\diff \Omega$ is defined
%
$$ \diff \Omega = sin(\theta) \diff \theta \diff \phi , $$
%
hence when integrated over a complete sphere, $\Omega = 4\pi$~sr.
 This corresponds to the area of a sphere with unit radius.
 Considering the fraction of the total surface area, $A$ of a sphere with radius $R$, we find that
%
\begin{align}
	\frac{A}{4\pi R^2} &= \frac{\Omega}{4\pi}
	\linedrop
	\Omega &= \frac{A}{R^2}.
	\label{eq:solid_angle}
\end{align}
%
As asserted previously, the solid angle corresponds to the area on a sphere with unit radius.
%
\subsubsection{Surface brightness}
Consider an object which a circular region in the sky with radius $\theta_c$.
 We choose to position this object such that it is located at the pole of the coordinate system.
 This is done for convenience such that the setup is rotationally symmetric in $\phi$.
 We can then calculate the flux incident on an aperture.
%
\begin{align*}
	F_\nu &= 2\pi I\nu \int_{0}^{\theta_c}\diff\theta \sin(\theta) \cos(\theta)
	\linedrop
	F_\nu &= \pi I_\nu \sin^2(\theta_c)
\end{align*}
%
For small $\theta_c$, $\sin(\theta_c) \simeq \tan(\theta_c) = R/r$ where $R$ is the object radius and $r$ is the object-aperture distance, hence
%
$$ F_\nu = \pi I_\nu  \left( \frac{R}{r} \right) ^2 \propto \frac{1}{r^2}. $$
%
For an isotropically emitting object (such as a star) the flux can be related to the luminosity (total power generated) by
%
$$ F = \frac{L}{4\pi r^2}. $$
%
%
%
\section{Radiative transfer}
\subsection{Free space}
We want to know how intensity, $I_\nu$ changes along the path of the ray.
 Consider two areas placed parallel to each other with a line length $R$ connecting their centres.
 If the areas are small relative to their separation, then all rays that pass through \emph{both} surfaces are effectively normal to the surfaces.
 From energy conservation,
 %
\begin{align*}
	\diff E_1 &= \diff E_2
	\linedrop
	I_{\nu ,1}\diff A_1 \diff t \diff \nu_1 \diff \Omega_1 &= I_{\nu ,2}\diff A_2 \diff t \diff \nu_2 \diff \Omega_2 ,	
\end{align*}
%
Since, in free space, the frequency remains constant and using equation \ref{eq:solid_angle}:
%
\begin{align*}
		I_{\nu ,1}\diff A_1 \frac{\diff A_2}{R^2} &= I_{\nu ,2}\diff A_1 \frac{\diff A_1}{R^2}
		\linedrop
		\implies I_{\nu ,1} &= I_{\nu ,2}.
\end{align*}
%
Given that the two surfaces have arbitrary size, we can conclude that in general, intensity is conserved along ray paths in free space.
 This is \emph{not} the case for flux.
 Intensity is conserved since it was defined as per unit solid angle.
 Hence, for a distance $s$ along the path of a ray,
%
$$ \frac{\diff I_\nu}{\diff s} = 0. $$
%
This is the equation of radiative transfer in free space.
 
 
 