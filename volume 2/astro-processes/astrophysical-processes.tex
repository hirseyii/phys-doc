\chapter{Astrophysical Processes}
\minitoc
\pagebreak
\section{Quantifying radiation}
\subsection{Flux \& intensity}
Recalling from \textit{Introduction to Astrophysics \& Cosmology} we define the energy flux density (or simply flux), $F$ as the power per unit surface area.
%
$$ dE = F dA dt $$
%
Note that flux is dependant on the orientation of the unit surface with respect to the Poynting vector of the radiation field.
 Flux has units of J/s/m$^2$.
 Often it is more useful to use the \emph{monochromatic} energy flux density, $F_\nu$ where
%
$$ dE = F_\nu dA dt d\nu. $$
%
This gives the energy flux per unit of bandwidth, (frequency interval of radiation) which has units J/s/m$^2$/Hz.
\par 
In order to account for the differing directions of photons, the flux is further generalised by defining the specific monochromatic intensity, $I_\nu$.
 This quantifies the power per unit frequency from a range of directions covering a solid angle element $d\Omega$ given by
%
$$ dE = I_\nu dA dt d\nu d\Omega. $$
%
This then has units J/s/m$^2$/Hz/sr.
 Unlike flux, the intensity is constant along the direction of propagation of photons (in free space).
 
\subsubsection{Solid angles: an aside}
The solid angle element $d\Omega$ is defined
%
$$ d\Omega = sin(\theta) d\theta d\phi , $$
%
hence when integrated over a complete sphere, $\Omega = 4\pi$~sr.
 This corresponds to the area of a sphere with unit radius.
 Considering the fraction of the total surface area, $A$ of a sphere with radius $R$, we find that
%
\begin{align*}
	\frac{A}{4\pi R^2} &= \frac{\Omega}{4\pi}
	\\[0.5\baselineskip]
	\Omega &= \frac{A}{R^2}.
\end{align*}
%
As asserted previously, the solid angle corresponds to the area on a sphere with unit radius.
 
 
 
 
 
 
 
 
 
 
 
 