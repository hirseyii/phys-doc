\chapter{Astrophysical Processes}
\minitoc
\pagebreak
\section{Quantifying radiation}
\subsection{Flux \& intensity}
Recalling from \textit{Introduction to Astrophysics \& Cosmology} we define the energy flux density (or simply flux), $F$ as the power per unit surface area.
%
$$ \diff E = F \diff A \diff t $$
%
Note that flux is dependant on the orientation of the unit surface with respect to the Poynting vector of the radiation field.
 Flux has units of J/s/m$^2$.
 Often it is more useful to use the \emph{monochromatic} energy flux density, $F_\nu$ where
%
\begin{equation}
	\diff E = F_\nu \diff A \diff t \diff \nu
	\label{eq:def_flux}
\end{equation}
%
This gives the energy flux per unit of bandwidth, (frequency interval of radiation) which has units J/s/m$^2$/Hz.
\par 
In order to account for the differing directions of photons, the flux is further generalised by defining the specific monochromatic intensity, $I_\nu$.
 This quantifies the power per unit frequency from a range of directions covering a solid angle element $\diff \Omega$ given by
%
\begin{equation}
	\diff E = I_\nu \diff A dt \diff \nu \diff \Omega.
	\label{eq:def_intensity}
\end{equation}
%
This then has units J/s/m$^2$/Hz/sr.
 Unlike flux, the intensity is constant along the direction of propagation of photons (in free space).
\par
From the respective definitions, we can relate flux and intensity by
%
\begin{align*}
	F_\nu \diff A &= \int \mathbf{I_\nu} \dotp \mathbf{\diff A} \diff \Omega
	\linedrop
	F_\nu \diff A &= \int I_\nu \cos(\theta) \diff A \diff \Omega
	\linedrop
	\therefore F_\nu &= \int I_\nu \cos(\theta) \diff \Omega
\end{align*}
%

\subsubsection{Solid angles: an aside}
The solid angle element $\diff \Omega$ is defined
%
$$ \diff \Omega = sin(\theta) \diff \theta \diff \phi , $$
%
hence when integrated over a complete sphere, $\Omega = 4\pi$~sr.
 This corresponds to the area of a sphere with unit radius.
 Considering the fraction of the total surface area, $A$ of a sphere with radius $R$, we find that
%
\begin{align}
	\frac{A}{4\pi R^2} &= \frac{\Omega}{4\pi}
	\linedrop
	\Omega &= \frac{A}{R^2}.
	\label{eq:solid_angle}
\end{align}
%
As asserted previously, the solid angle corresponds to the area on a sphere with unit radius.
%
\subsubsection{Surface brightness}
Consider an object which a circular region in the sky with radius $\theta_c$.
 We choose to position this object such that it is located at the pole of the coordinate system.
 This is done for convenience such that the setup is rotationally symmetric in $\phi$.
 We can then calculate the flux incident on an aperture.
%
\begin{align*}
	F_\nu &= 2\pi I\nu \int_{0}^{\theta_c}\diff\theta \sin(\theta) \cos(\theta)
	\linedrop
	F_\nu &= \pi I_\nu \sin^2(\theta_c)
\end{align*}
%
For small $\theta_c$, $\sin(\theta_c) \simeq \tan(\theta_c) = R/r$ where $R$ is the object radius and $r$ is the object-aperture distance, hence
%
$$ F_\nu = \pi I_\nu  \left( \frac{R}{r} \right) ^2 \propto \frac{1}{r^2}. $$
%
For an isotropically emitting object (such as a star) the flux can be related to the luminosity (total power generated) by
%
$$ F = \frac{L}{4\pi r^2}. $$
%
%
%
\section{Radiative transfer}
\subsection{Free space}
We want to know how intensity, $I_\nu$ changes along the path of the ray.
 Consider two areas placed parallel to each other with a line length $R$ connecting their centres.
 If the areas are small relative to their separation, then all rays that pass through \emph{both} surfaces are effectively normal to the surfaces.
 From energy conservation,
 %
\begin{align*}
	\diff E_1 &= \diff E_2
	\linedrop
	I_{\nu ,1}\diff A_1 \diff t \diff \nu_1 \diff \Omega_1 &= I_{\nu ,2}\diff A_2 \diff t \diff \nu_2 \diff \Omega_2 ,	
\end{align*}
%
Since, in free space, the frequency remains constant and using equation \ref{eq:solid_angle}:
%
\begin{align*}
		I_{\nu ,1}\diff A_1 \frac{\diff A_2}{R^2} &= I_{\nu ,2}\diff A_1 \frac{\diff A_1}{R^2}
		\linedrop
		I_{\nu ,1} &= I_{\nu ,2}.
\end{align*}
%
Given that the two surfaces have arbitrary size, we can conclude that in general, intensity is conserved along ray paths in free space.
 This is \emph{not} the case for flux.
 Intensity is conserved since it is defined as `per unit solid angle'.
 Hence, for a distance $s$ along the path of a ray,
%
$$ \frac{\diff I_\nu}{\diff s} = 0. $$
%
This is the equation of radiative transfer in free space.
%
%
\subsection{In a medium}
We now need  to consider emission generated along the line of sight as well as absorption by the medium.
 This can be quantified by the equation of radiative transfer in a medium,
%
\begin{equation}
	\frac{\diff I_\nu}{\diff s} = j_\nu - \alpha_\nu I_\nu,
	\label{eq:radiative_transfer}
\end{equation}
%
where $j_\nu$ is the \emph{monochromatic emission coefficient} with units J/s/m$^3$/Hz/sr, and $\alpha_\nu$ is the \emph{monochromatic absorption constant} with units m$^{-1}$.
 It is simple to realise that doubling the intensity passing through a medium would in turn double the intensity absorbed per metre.
 Note that in free space, $j_\nu = 0$ and $\alpha_\nu = 0$.
\par
\subsubsection{Emission coefficient}
From the definition of intensity given by equation \ref{eq:def_intensity}, but changing $\diff E$ to $\delta_E$,
%
\begin{align}
	\frac{\diff \delta_E}{\diff s} &= \frac{\diff I_\nu}{\diff s} \diff A \diff t \diff \nu \diff \Omega
	\linedrop
	\frac{\diff \delta_E}{\diff s} &= j_\nu \diff A \diff t \diff \nu \diff \Omega
	\linedrop
	\therefore \diff \delta_E &= j_\nu \diff V \diff t \diff \nu \diff \Omega.
	\label{eq:emission_coef}
\end{align}
%
From equation \ref{eq:emission_coef} we can see that $j_\nu$ corresponds to the amount of energy radiated per unit volume\footnote{Note that $\diff V = \diff A \diff s$.}, per unit time, per unit solid angle for a frequency interval.
\par 
\subsubsection{Absorption constant}
By using similar treatment for the absorption, we find that
%
\begin{align}
		\frac{\diff \delta_E}{\diff s} &= -\alpha_\nu I_\nu  \diff A \diff t \diff \nu \diff \Omega
		\linedrop
		\frac{\diff \delta_E}{\diff s} &= -\alpha_\nu \delta_E.
		\label{eq:absorption_const}
\end{align}
%
Equation \ref{eq:absorption_const} implies that $\alpha_\nu \diff s$ represents the fraction of energy that disappears from the radiation field after travelling a length $\diff s$ through the medium.
\par  
\subsubsection{Opacity}
Following from the definition in equation \ref{eq:absorption_const}, we now consider a volume of absorbing particles with number density $n$ and absorption cross section $\sigma_\nu$.
 Since each particle obscures an area $\sigma_\nu$, the total area blocked is $n\sigma_\nu\diff V$.
 Also, the fractional area (through which photons flow) that is blocked by absorbers is
$$ n \sigma_\nu \frac{\diff V}{\diff A} = n \diff s \sigma_\nu, $$
hence
% 
\begin{align*}
	\alpha_\nu &= n \sigma_\nu	
	\linedrop
	\alpha_\nu &= \rho \kappa_\nu.
\end{align*}
%
The \emph{opacity coefficient} $\kappa_\nu$ is defined such that $\sigma_\nu \propto \kappa_\nu$. Also, $\kappa_\nu$ has units m$^2$/kg.
\par 
\subsubsection{Optical depth}
A useful quantity to define is optical depth $\tau_\nu$, which is a re-scaled distance co-ordinate defined by
$$ \diff \tau_\nu \equiv \alpha_\nu \diff s, $$
which provides a more relevant unit to express distance in, rather than metres.
 The total optical depth can then be defined as
$$ \tau_\nu (s) = \int_{s_0}^{s} \alpha_\nu \diff s^\prime. $$
The optical depth becomes particularly useful in the case $j_\nu = 0$.
 In this situation, the equation of radiative transfer (see equation \ref{eq:radiative_transfer}) becomes
$$ \diff I_\nu = - I_\nu \diff \tau_\nu. $$
This differential equation is easy to solve and we find that
%
\begin{equation}
I_\nu (\tau_\nu) = I_{\nu,0} e^{\tau_\nu}.
\end{equation}
%
\par 
\subsubsection{Source function}
Another useful quantity to define is the source function which follows a similar derivation. %REF Draine 7.4
This, along with the definition for optical depth can be used to further generalise the equation of radiative transfer (\ref{eq:radiative_transfer}).
 The source function, $S_\nu$ is defined as
%
\begin{equation}
	S_\nu \equiv \frac{j_\nu}{\alpha_\nu}.
\end{equation}
%
The radiative transfer equation can now be defined using both $S_\nu$ and $\tau_\nu$ as
%
\begin{align*}
	\diff I_\nu &= \frac{j_\nu}{\alpha_\nu} \diff \tau_\nu - I_\nu \diff \tau_\nu
	\linedrop
	\diff I_\nu &= S_\nu \diff \tau_\nu - I_\nu \diff \tau_\nu.
\end{align*}
%
We can then integrate this to obtain the solution by introducing an integration factor $e^{\tau_\nu}$.
%
\begin{align*}
	\frac{\diff I_\nu}{\diff \tau_\nu} + I_\nu &= S_\nu
	\linedrop
	e^{\tau_\nu} \left( \frac{\diff I_\nu}{\diff \tau_\nu} + I_\nu \right) &= e^{\tau_\nu} S_\nu
	\linedrop
	\text{product rule} \implies 
	\frac{\diff}{\diff \tau_\nu} \left( e^{\tau_\nu} I_\nu \right) &= e^{\tau_\nu} S_\nu
	\linedrop
	\diff \left( e^{\tau_\nu} I_\nu \right) &= e^{\tau_\nu} S_\nu \diff \tau_\nu
\end{align*}

