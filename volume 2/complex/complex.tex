\chapter{Complex variables and vector spaces}
\minitoc
\pagebreak
\section{Complex Numbers revision \& mappings}
\subsection{Basics}
We start with a basic revision of complex numbers.
 Complex numbers were initially introduced to ensure that every polynomial of degree n has n roots. 
 A complex number, $z$, can be expressed in 3 forms:
\begin{enumerate}
	\item $z=x+iy$
	\item $z=R(cos\theta+isin\theta)$
	\item $z=Re^{i\theta}$
\end{enumerate}
In case your memory is that poor, $i$ is the square root of -1.
 Addition of complex numbers is defined as follows $$(x+iy) + (a+ib) = (x+a) + i(y+b) $$
which should be pretty obvious.
 Subtraction is not defined explicitly but by changing $a$ to $-a$ and $b$ to $-b$.
  Multiplication is also defined $$(x+iy)(a+ib) = (ax-by)+i(bx+ay)$$
\subsection{Argand Diagrams}
An argand diagram is a way of representing a complex number on a 2D plane.
 The real component of $z$ is plotted on the x-axis whilst the imaginary component is plotted on the y-axis.
 %
 \begin{minipage}[t]{0.47\linewidth}
 	\begin{figure}[H]
 		\centering
 		\includegraphics[width=\linewidth]{complex/argand}
 		\captionsetup{font=small} 	
 	\end{figure} 
 \end{minipage}
 \hspace{0.6cm}
%
\begin{minipage}[t]{0.47\linewidth}
	\vspace{1cm}
	We can define the modulus 
	%
	\begin{align*}
	\mid z \mid &= \sqrt{x^2+y^2} \\
	&= (x+iy)(x-iy) \\
	&= z\overline{z}.
	\end{align*}
	%
	In addition to this, $\theta$ is often referred to as the argument.
	 Note that the argument is not unique, you can add $2\pi$ to the argument and obtain the same result, so it's often useful to think of the principal value argument adding the contraint that $\theta$ lies between $0$ and $2\pi$
\end{minipage}
\begin{examples}
	We'll start with a few basic examples to refresh your memory, given $z_1=1+i$ \& $z_2=2+3i$:
	\begin{enumerate}
		\item Find the cube root of $z_1$
		\item Find the modulus and argument of $z_1z_2$
	\end{enumerate}
\textbf{Answers:} 1. $2^{1/6}exp(\frac{\pi}{12})$, $2^{1/6}exp(\frac{3\pi}{4})$, $2^{1/6}exp(\frac{17\pi}{12})$ \hspace{0.5cm}
2. $\mid z\mid = \sqrt{26}$, $arg(z)=1.77$
\end{examples}
\subsection{Mappings}
If we have a complex function, $w=f(z)$, then you can think of this function as a mapping from the domain, the $z$ complex plane, to the co-domain, the $w$ complex plane.
 For example, consider $w=z^2$, what does this mapping look like?
  If we assume $w=u(x,y)+iv(x,y)$ we can see that $u=x^2-y^2$ \& $v=2xy$.

\begin{minipage}[t]{0.47\linewidth}
	\begin{figure}[H]
		\centering
		\includegraphics[width=\linewidth]{complex/mapping}
		\captionsetup{font=small} 	
	\end{figure} 
\end{minipage}
\hspace{0.6cm}
%
\begin{minipage}[t]{0.47\linewidth}
	\vspace{0.3cm}
	With a little bit of magic algebra, by setting $x=a$ and $y=b$ we can get 2 relationships.
	\begin{align*}
	\frac{v^2}{4a^2}=a^2-u \\
	\frac{v^2}{4b^2}=u+b^2. \\
	\end{align*}
These 2 equations are plotted on the left for different values of $a$ and $b$.
 Remembering that $a$ and $b$ are $x$ and $y$ respectively then the red lines are how lines of constant x in the z-plane map into the w-plane.
\end{minipage}
\subsection{The argument theorem}
Consider a function $f(z)$, if $f(z_1)=0$ then $z_1$ will map onto the point $w=0$ in the w-plane.
 The argument theorem states that if a contour in the z-plane wraps around n roots of $f(z)$ then it's mapping will wrap around $w=0$ n times. 
 This is illustrated below:
\begin{figure}[H]
	\centering
	\includegraphics[width=\linewidth]{complex/argthm}
	\captionsetup{font=small} 	
\end{figure}
\noindent Another statement of the argument theorem is that along the path, $C$, the argument of w changes by $2n\pi$ if n zeroes are enclosed by $C$.