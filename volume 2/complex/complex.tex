\chapter{Complex variables and vector spaces}
\minitoc
\pagebreak
\section{Complex Numbers revision}
\subsection{basics}
We start with a basic revision of complex numbers. Complex numbers were initially introduced to ensure that every polynomial of degree n has n roots. A complex number, $z$, can be expressed in 3 forms:
\begin{enumerate}
	\item $z=x+iy$
	\item $z=R(cos\theta+isin\theta)$
	\item $z=Re^{i\theta}$
\end{enumerate}
In case your memory is that poor, $i$ is the square root of -1. Addition of complex numbers is defined as follows $$(x+iy) + (a+ib) = (x+a) + i(y+b) $$
which should be pretty obvious. Subtraction is not defined explicitly but by changing $a$ to $-a$ and $b$ to $-b$. Multiplication is also defined $$(x+iy)(a+ib) = (ax-by)+i(bx+ay)$$
\subsection{Argand Diagrams}
An argand diagram is a way of representing a complex number on a 2D plane. The real component of $z$ is plotted on the x-axis whilst the imaginary component is plotted on the y-axis. 



